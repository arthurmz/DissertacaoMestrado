% Resumo em l�ngua vern�cula
\begin{center}
	{\Large{\textbf{T�tulo do trabalho}}}
\end{center}

\vspace{1cm}

\begin{flushright}
	Autor: Jos� Arthur Souza de Mac�do\\
	Orientador(a): Titula��o e nome do(a) orientador(a)
\end{flushright}

\vspace{1cm}

\begin{center}
	\Large{\textsc{\textbf{Resumo}}}
\end{center}

\noindent O Hiker Dice � um jogo l�gico onde um determinado dado reside sobre um tabuleiro de nxm casas. O dado assim pode se mover entre as casas por meio de uma opera��o de rolagem que quando aplicada, determina a pontua��o que o jogador ganha, correspondente ao valor da nova face do dado que toca a casa. A rolagem para uma casa marca a casa visitada, assim nenhuma pode ser repetida. O jogo se encerra quando n�o existem novas casas dispon�veis para rolar. Neste trabalho � tratado o problema do Hiker Dice Hamiltoniano, que � caracterizado como o problema de otimiza��o de um ciclo hamiltoniano sobre um grafo grade com "buracos". Tal ciclo � ponderado em fun��o dos pontos ganhos a cada jogada do dado do Hiker Dice. 

\noindent\textit{Palavras-chave}: Palavra-chave 1, Palavra-chave 2, Palavra-chave 3.